\documentclass[a4paper]{article}
\usepackage{fancyhdr}

\input{setup.tex}

%--------------------------------------- DÉBUT DU DOCUMENT ----------------------------------------%

\begin{document}

%Page de garde
\begin{titlepage}
	\enlargethispage{2cm}

	\begin{center}

		\textsc{\@title}
		\vspace*{0.5cm}

		\large{\@author}

		\vspace*{0.5cm}
		\large{\LARGE{\textbf{\Subject}}}
		
		\Dates

		\vspace*{0.5cm}
		\includegraphics[scale=0.5]{data/Thales_LOGO.png}

		\vspace*{1cm}
		\includegraphics[scale=0.75]{data/ESEO_LOGO.pdf}

	\end{center}

	\vfill

	\begin{tabular}{llll}
		\textsc{\textbf{option}} 
			& $\square$ EOC & $\boxtimes$ LEC & $\square$ ISI \\
		\textsc{\textbf{confidentialité}} 
			& \multicolumn{3}{l}{Mon rapport est confidentiel $\square$ 1 (moy) $\square$ 2 (élevé)} \\
		\textsc{\textbf{domaine entrprise}}	
			& \multicolumn{3}{l}{$\boxtimes$ Cybersécurité, data science et intelligence artificielle} \\
			& \multicolumn{3}{l}{$\square$ Véhicules intelligents} \\
			& \multicolumn{3}{l}{$\square$ Santé} \\
			& \multicolumn{3}{l}{$\square$ Développement durable et villes intelligente}\\
			& \multicolumn{3}{l}{$\square$ Industrie du futur} \\
			& \multicolumn{3}{l}{$\square$ Ingénierie d'affaires} \\		
	\end{tabular}

	\vspace*{0.75cm}

	\SupervisorSection

	\TutorSection
	\vspace{2cm}

\end{titlepage}

%Paramétrage de l'en-tête et du pied-de-page
\fancyhead[L]{\includegraphics[scale=0.2]{data/ESEO_LOGO.pdf}}
\fancyhead[R]{\includegraphics[scale=0.15]{data/Thales_LOGO.PNG}}
\fancyhead[C]{\@title}
\fancyfoot[L]{\SemesterSection}
\fancyfoot[R]{\CycleSection}
\fancyfoot[C]{\thepage}

\newpage
\large{\textsc{\textbf{Engagement de non-plagiat}}}
\vspace*{0.5cm}

Je soussigné, \Author, étudiant à l'ESEO, atteste avoir pris connais-
sance du contenu du Règlement intérieur de l'École et de l'engagement de « non-plagiat
». Je déclare m'y conformer dans le cadre de la rédaction de ce document. Je déclare sur
l'honneur que le contenu du présent mémoire est original et reflète mon travail personnel.
J'atteste que les citations sont correctement signalées par des guillemets et que les sources
de tous les emprunts ponctuels à d'autres auteur(e)s, textuels ou non textuels, sont indi-
quées. Le non-respect de cet engagement m'exposerait à des sanctions dont j'ai bien pris
connaissance.

Fait à ......... le .......

Signature

\newpage
\large{\textsc{\textbf{Remerciements}}}
\vspace*{0.5cm}

\newpage
\vspace*{1cm}
\tableofcontents

\newpage
\large{\textsc{\textbf{Fiche de synthèse}}}
\vspace*{0.5cm}

\textbf{Sujet} : \Subject

\textbf{Entreprise} : Thales SIX GTS, 49300 Cholet, France

\textbf{Dates du projet} : 1er septembre 2024 au 28 août 2025

\textbf{Collaborateurs} : Jean-Christophe DUBOIS, Bruno LETELLIER \\\\
\textbf{Résumé} : problématique objectifs au format SMART, planning synthétique (réalisations PERSONNELLES de l'apprenti), démarche pour résoudre le problème, résultats quantitatifs en lien avec les objectifs, conclusion et implications pour l'entreprise.

\newpage
\large{\textsc{\textbf{Abstract}}}
\vspace*{0.5cm}

The abstract is NOT a summary of your report, it should work as a teaser with a fix
structure :

1) Starting point -Motivation (Why do we care about the problem and the results ?)

2) Questioning = Key question (What problem are you trying to solve ? What is the scope
of your work ?)

3) Research objectives ?

4) Process / Approach / implementation (How did you go about solving or making pro-
gress on the problem ? Did you use simulation, analytic models, prototype construction,
or analysis of field data for an actual product or service, etc. ?)

5) Results/Outcomes - When presenting results, you should give enough information
about how the findings were obtained so that the broad outlines of your reasoning can
be easily understood without disclosing too much detail about your research. (What is the
answer ? What are the exact results/figures ?)

6) Conclusions - closing sentences with hypotheses and implications (What are the impli-
cations of your answer ? Are your results general, potentially generalizable, or specific to
a particular case ?)

\newpage
\section{Introduction}

\newpage
\section{Contexte}

\subsection{Service, position de l'apprenant}

\subsection{Contexte du projet}

\subsection{Contexte technique ou métier}

\subsection{Impacts sociaux-écologiques}

\newpage
\section{Problématique}

\subsection{Problématique spécifique}

\subsection{Personnes impactées}

\subsection{Quantification des impacts}

\newpage
\section{Buts et objectifs}

\subsection{Buts et objectifs SMART}

\subsection{Gains et coûts attendus}

\subsection{Identification des acteurs}

\subsection{Analyse des risques / SWOT}

\newpage
\section{Démarche}

\subsection{Démarche générale}

\subsubsection{Démarche générale}

\subsubsection{Méthodologie, techniques et technologies}

\subsection{Lotissement}

\subsubsection{Lotissement cohérent}

\subsection{Planning prévisionnel}

\subsubsection{Planning prévisionnels}

\newpage
\section{Conclusion}

\newpage
\section{Bibliographie}

\newpage
\section{Glossaire}

\newpage
\appendix

\section{Annexes}
\end{document}